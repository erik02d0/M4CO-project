\documentclass[a4paper,11pt]{article}

\usepackage{amsmath}
\usepackage{amssymb}
%\usepackage{fullpage}

\title{Model Description}

\begin{document}
\maketitle

\noindent
The problem is as follows: Divide $n$ workers up into $t$ teams so that each team contains at least one driver, and "worker satisfaction" is maximized (we will define the latter term further down).

The parameters for this problem are, of course, the number of workers, the number of teams (number of workers per team is $\texttt{wpt}=n/t$) and the number of drivers $d\geq t$. In addition to this we suppose that we have a "preference matrix" $P=(p_{ij}) \in \mathbb{Z}^{n\times n}$. Each entry $0\leq p_{ij}\leq n-1$ denotes worker $i$'s preference to work in the same team as worker $j$, where a $1$ means most preferred and $n-1$ means least preferred. Each worker "prefers" himself the most, whence $p_{ii}=0\ \forall i$. Preference is not reciprocal, so $p_{ij}$ and $p_{ji}$ need not be equal. We're assuming that each person values all his coworkers differently, i.e no two elements in any row or column are the same.

Getting an instance of $P$ requires some manual work, as MiniZinc will generate a rather artificial instance from the above constraints.

We can view the workers simply as an array $(1,...,n)$ where we let the first $d$ elements be the drivers. The workers are divided into teams. Such a division is called \texttt{shift} (as in "the division for today's shift") and is simply a permutation of $(1,...,n)$. From among the $d$ drivers we choose $t$ drivers, one for each team, held in \texttt{teamDrivers}. The driver for team $1$ is \texttt{teamDrivers[1]}, the driver for team $2$ is \texttt{teamDrivers[2]}, and so on. A possible symmetry-breaking constraint is to require that \texttt{teamDrivers}, viewed as a sequence, is increasing. For example, if $(n,t,d)=(4,2,2)$ then obviously the shifts $(1,3,2,4)$ and $(2,4,1,3)$ are equivalent, and this possibility can be excluded by using said constraint.

Each team's driver is put in shift position \texttt{wpt*(i-1)}, $i=1,...,t$ and the remaining team members can be thought of as occupying positions $\texttt{wpt*(i-1)+2},...,\texttt{wpt*(i-1)+wpt}$.\newline
\newline
\noindent
At this point we can begin calculating the "worker satisfaction" mentioned above. (to be continued...)

\end{document}
