\documentclass[a4paper, 11pt]{article}

\usepackage{amsmath}
\usepackage{amssymb}

\title{M4CO Project}
\author{Erik Vesterlund}

\begin{document}
\maketitle

Let $n$ be the number of employees, $t$ the team size, and $d$ the number of drivers. We demand that $n\in \mathbb{Z}/(t)$ and $d\geq n/t$. Introduce an $n\times n$ preference matrix $P = (p_{ij})$ and construct the entries as follows: if person $i$'s most preferred co-worker is $j$ then $p_{ij}=1$, if $j$ is $i$'s second most preferred co-worker then $p_{ij}=2$, and so on. We should set $p_{ii}=0$ for all $i$, and we should also assume that level of preference is not necessarily requited, i.e. $P$ is not necessarily symmetric. What we want to ultimately do is construct a partition of the shift such that 1) there's at least one driver in each team, and 2) worker satisfaction is "optimal".

Given a preference matrix $P$ and a permutation $\sigma \in S_n$ we could let $\sigma P$ be a shift partition, where $\sigma$ permutes the columns (or rows) of $P$ and say that columns $1,...,t$ constitute one team, columns $t+1,...,2t$ another, and so on. Naturally, since the members of a team are not ordered within the team (there is no "captain") there are several permutations $\sigma_i,\sigma_j$ such that $\sigma_i P,\sigma_j P$ represent the same shift partition. Anyway, for the first team we'd be interested in $p_{ij}$ for $i,j=1,...,t$, $i\neq j$...



\end{document}
